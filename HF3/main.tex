\documentclass{article}
\usepackage{graphicx} % Required for inserting images
\usepackage[a4paper, total={7in, 10in}]{geometry}
\usepackage{booktabs}

\title{Lakásárak becslése regressziós modellezéssel}
\author{Molnár Marcell, CEZCRR - marcell.molnar@stud.uni-corvinus.hu}
\date{}

\begin{document}
\maketitle
\pagenumbering{arabic}
\begin{center}    
Ebben a projektfeladatban lakásárak előrejelzésére építünk különböző komplexitású regressziós modelleket. A modellek célja az előrejelzés, ezért mindhárom modellértékelési kritériumban valamilyen formában negatívan fogjuk súlyozni a komplexitást a túltanulás elkerülése érdekében. A kiválasztott modelleket a Test halmazon fogjuk összehasonlítani.
\end{center}
\section{Előkészületek, adattisztítás}
A magyarázó változók kombinációinak segítségével 8 eltérő komplexitású modell kerül definiálásra. A legjobb predikciós modell megtalálása érdekében 3 különböző módszer alapján választjuk ki külön-külön a legjobb előrejelző képességűeket (direkt módszer, indirekt módszer, Lasso regularizáció). A minta (545 megfigyelés, 12+1 magyarázó változó) magyarázó változói között találhatók számszerű adatok (alapterület, hálószobák száma, fürdőszobák száma, emeletek száma, parkolóhelyek száma), bináris változók (főút melletti, vendégszoba, légkondicionálás, vízfűtés, népszerű fekvés) és egy bináris kódolású változó is a bútorozásra.
\\
\\
Alkotott odellek a magyarázó változóik alapján: \\
\begin{table}[h!]
    \centering
    \begin{tabular}{|c|c|c|r|}
    \toprule
     & Model ID & Size & Independent variables \\
    \midrule
    0 & 1 & 1 & area \\
    1 & 2 & 1 & parking \\
    2 & 3 & 2 & stories+prefarea \\
    3 & 4 & 3 & area+stories+prefarea \\
    4 & 5 & 4 & area+stories+bathrooms+prefarea \\
    5 & 6 & 4 & area+bedrooms+mainroad+guestroom \\
    6 & 7 & 5 & area+hotwaterheating+airconditioning+parking+prefarea \\
    7 & 8 & 13 & összes magyarázó változó \\
    \bottomrule
    \end{tabular}
    \caption{Modelldefiníciók}
\end{table}
\\
A modellek között szerepel 2 darab egyváltozós modell, 5 darab "alacsony komplexitású" (2-4 magyarázó változó), egy 5 változós és az összes magyarázó változót tartalmazó is.
\section{Módszertan}
Az adatokat 80-20 százalékban véletlenül felbontjuk Train és Test halmazokra, ahol kibecsüljük a coefficiensek értékeit, és összehasonlítjuk a modellek teljesítményét. \\
Az összehasonlítást 3 módon végezzük:
\begin{itemize}
    \item Direkt módszer: Az adathalmazt tovább osztjuk 80-20 százalékban Train és Validation halmazokra, ahol a Train halmazon végezzük a paraméterbecslést, és a Validation halmazon számított RMSE alapján választjuk ki a legjobb modellt
    \item Indirekt módszer: A modellek kiválasztása az információs kritériumok alapján történik, külön az AIC és külön a BIC szerint is
    \item Lasso regularizáció: A kiválasztás a paraméterbecsléssel egyidőben történik, ahol különböző lambda paraméterrel állítjuk be a modell komplexitásának büntetését
\end{itemize}
Először csak egyszer végezzük el az adatok szétválasztását, és értelmezzük a kapott eredményeket, majd megismételjül 100-szor szimuláció segítségével is.
\newpage
\section{Elsőkörös eredmények}
\subsection{Direkt módszer}
Egyszeri szétosztás után a direkt módszer a 8-adik, legkomplexebb modellt értékeli a Validation RMSE alapján a legjobbnak, viszont látható, hogy akár az 5-ös, akár a 6-os számú modell hasonló hibával becsül a Validation halmazon. Ez ellentmondhat a túltanulás állításainak, de az is lehet, hogy csak az adatok véletlen szétosztásának műve.
\begin{figure}[h!]
    \centering
    \includegraphics[width=0.75\linewidth]{direkt.pdf}
    \caption{Direkt modell RMSE értékei}
\end{figure}
\subsection{Indirekt módszer}
Az indirekt módszer esetén a Train halmazt nem bontjuk tovább két részre, hanem az Akaike (AIC) és Bayesian (BIC) információs kritérium alapján rangsoroljuk őket. Az információs kritériumok büntetik modell komplexitását, viszont meglepő módon ebben az esetben is a legkomplexebb modell került ki győztesnek.
\begin{table}[h!]
    \centering
    \begin{tabular}{|c|c|c|c|c|c|c|c|c|}
    \toprule
    Place & Model ID & IC & Size & --------- &Place & Model ID & IC & Size \\
    \midrule
    1 & 8 & AIC & 13 & & 1 & 8 & BIC & 13 \\
    2 & 5 & AIC & 4 & & 2 & 5 & BIC & 4 \\
    3 & 7 & AIC & 5 & & 3 & 7 & BIC & 5 \\
    4 & 4 & AIC & 3 & & 4 & 4 & BIC & 3 \\
    5 & 6 & AIC & 4 & & 5 & 6 & BIC & 4 \\
    6 & 1 & AIC & 1 & & 6 & 1 & BIC & 1 \\
    7 & 3 & AIC & 2 & & 7 & 3 & BIC & 2 \\
    8 & 2 & AIC & 1 & & 8 & 2 & BIC & 1 \\
    \bottomrule
    \end{tabular}
    \caption{AIC és BIC rangsorok}
\end{table}
\\
Ezek az eredmények már érdekesebb lehetnek számunkra. Mivel ebben az esetben is a legkomplexebb modellt kaptuk meg, mint legjobb előrejelző modell, elképzelhető, hogy a modelldefinícióinkkal. Látszik, a direkt módszer alapján is, hogy akár egy 4 magyarázó változójú modell is hasonló RMSE-t tud produkálni a Validation halazon, szóval elképzelhető az is, hogy rossz modelleket definiáltunk.
\subsection{Lasso regularizáció}
A Lasso regularizáció eljárás alapfeltétele az, hogy a magyarázó változókat "leszűkítse", akár 0-ra redukálja a modellben. Az eljárást a legbővebb modellre futtatjuk. A kiválasztott alpha 0-körüli (mintha rendes OLS becslés lenne), amiből arra tudunk következtetni, hogy a legjobb előrejelző modellhez sok magyarázó változóra van szükségünk.
\newpage
\section{Szimulációk}
\subsection{Stabilitás}
Az adatok véletlenszerű felosztását most 100-szor fogjuk elvégezni, hogy egy "stabil" legjobb modellt tudjuk kiválasztani. A direkt módszert és a Lasso regularizációt továbbá keresztvalidációval fogjuk számolni.
\begin{figure}[h!]
    \centering
    \includegraphics[width=1\linewidth]{stability.pdf}
    \caption{Modellválasztás stabilitása}
\end{figure}
\\
100-szoros véletlen felosztás után is látható, hogy szinte mindig a 8-as, legkomplexebb modellt választjuk mind a direkt és indirekt módszer alapján.
\subsection{Holdout teljesítmény}
Végső teljesértékelésként a szimuláció során a Test halazra előrejelzéseket végzünk a kiválaszott modellekkel, és ábrázoljuk boxploton az RMSE értékeket. Látható, hogy az átlagok és a kvartilisek is hasonlóan alakulnak egymáshoz.
\begin{figure}[h!]
    \centering
    \includegraphics[width=0.8\linewidth]{holdout.pdf}
    \caption{Holdout teljesítmény}
\end{figure}
\section{Értékelés}
A szimulációk során szinte minden esetben a legkomplexebb modellünk lett a legjobb előrejelző. Ez azt sugallhatja, hogy egy lakás beárazásához az adott magyarázó változók szinte mindegyike szükséges, azaz mindegyikük hordoz magában új információkat (esetleg legondolkozhatunk a multikollinearitás vizsgálatán is).
\\
\\
Viszont néhány esetben a szimuláció során a 5-ös modell is kiválasztára került (kizárólag a direkt módszer alapján), ami azt sugallhatja, hogy lehetséges egy olyan modellt építeni, ami a legjobb előrejelző képességgel rendelkezik, de mégsem a legkomplexebb. Az 5-ös modell például csak az alapterületet, a fürdők számát, az emeletek számát és a preferált fekvést tartalmazza, mint magyarázó változót.
\\
Egy lehetséges továbbvitel az lehet, hogy megépítjük az összes több, mint 10 magyarázó változót tartalmazó komplex modellt, és ezeket vetjül össze egymáshoz képest.
\section*{Melléklet}
\begin{table}[h!]
    \centering
    \begin{tabular}{|c|c|c|c|c|}
        \hline
          & Direkt RMSE & Indirekt AIC és BIC RMSE & Lasso CV RMSE & Lasso $\lambda$\\
         \hline
         Átlag & 1097324.97 & 1096453.78 & 1096704.21 & 0.001521 \\
         Szórás & 112776.17 & 111817.84 & 112608.01 & 0.001705\\
         \hline
    \end{tabular}
    \caption{Holdout átlag RMSE 100 szimulációra}
\end{table}
\begin{table}[h!]
    \centering
    \begin{tabular}{|c|c|c|c|c|}
    \toprule
     & Model ID & Set & RMSE & Size \\
    \midrule
    0 & 1 & Train & 1807551.37 & 1 \\
    1 & 2 & Train & 3655572.72 & 1 \\
    2 & 3 & Train & 1974737.18 & 2 \\
    3 & 4 & Train & 1366788.10 & 3 \\
    4 & 5 & Train & 1218600.66 & 4 \\
    5 & 6 & Train & 1396699.33 & 4 \\
    6 & 7 & Train & 1537834.73 & 5 \\
    7 & 8 & Train & 1030802.25 & 13 \\
    8 & 1 & Validation & 1853442.64 & 1 \\
    9 & 2 & Validation & 3599062.57 & 1 \\
    10 & 3 & Validation & 2096772.02 & 2 \\
    11 & 4 & Validation & 1463575.88 & 3 \\
    12 & 5 & Validation & 1287483.27 & 4 \\
    13 & 6 & Validation & 1471958.40 & 4 \\
    14 & 7 & Validation & 1588596.61 & 5 \\
    15 & 8 & Validation & 1173102.82 & 13 \\
    \bottomrule
    \end{tabular}
    \caption{Direkt módszer Train-Validation RMSE}
\end{table}
\begin{table}[h!]
    \centering
    \begin{tabular}{|c|c|c|c|c|c|c|}
    \toprule
    Konstans & 'area' & 'bedrooms' &'bathrooms' & 'stories' & 'mainroad' & 'guestroom' \\
    -4.57e+05 &  2.17e+02 & 1.90e+05 & 9.23e+05 & 4.40e+05 & 5.94e+05 & 1.78e+05 \\
    basement' & 'hotwaterheating' & 'airconditioning' & 'parking' & 'prefarea' & 'dfurnished' & 'dsemifurnished' \\
    3.93e+05 & 6.41e+05 & 9.52e+05 & 2.77e+05 & 5.89e+05 & 3.88e+05 & 2.40e+05 \\
    \bottomrule
    \end{tabular}
    \caption{Direkt modell béta értékeinek átlaga (100-szoros szimuláció)}
\end{table}
\end{document}
